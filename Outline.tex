
\documentclass[twoside]{Style/ucasthesis}%

\usepackage[authoryear,list]{Style/artratex}% document settings

\usepackage{Style/artracom}% user defined commands

\begin{document}

\mainmatter% initialize the environment
\chapter{引言}
\section{研究背景}
介绍图计算
\section{LazyGraph 存在的问题}
lazygraph 的性能提升效果依赖于具体的延迟一致开启策略
lazygraph 现在采用一个决策树模型作为开启策略,它存在着以下这些方面的不足
1. 难以解释
2. 训练过程依赖于手动调优
3. 无法做到自适应
\section{基于解的关系的自适应优化方法}
通过实验发现, 解存在着局部性规律可以作为自适应开启策略
\section{本文的结构概述}
每一章写了什么

\chapter{相关技术介绍及研究动机}
\section{分布式并行图处理框架}
\subsection{任务调度机制}
\subsection{通信机制}
\subsection{图划分}
\subsection{计算粒度与顶点计算模型}

\section{基于延迟数据一致性方法的 LazyGraph 图计算系统}
\subsection{顶点副本的延迟数据一致性方法}
延迟数据一致方法是什么
\subsection{延迟数据一致性方法的开启策略}
现有的开启策略,及不足之处


\chapter{基于解的局部性的自适应优化方法}
\section{LazyGraph 性能提升规律研究}
有效计算与无效计算
\section{全局解和局部解的关系规律分布}
解的局部性
\section{基于全局解和局部解的关系的自适应优化方法}
在线统计,在线自适应



\chapter{实验结果}

\chapter{总结与分析}


%-
%-> Appendix
%-
\cleardoublepage%
\end{document}
%---------------------------------------------------------------------------%
