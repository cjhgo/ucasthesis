%---------------------------------------------------------------------------%
%->> Backmatter
%---------------------------------------------------------------------------%
\chapter{作者简历及攻读学位期间发表的学术论文与研究成果}



\section*{作者简历}

陈军航,男,河南省许昌市鄢陵县人,
2012年—2016年在江苏科技大学 就读计算机科学与技术专业并获得学士学位,
2016年—2017年在在南京魔格信息技术有限公司工作,
2017年—2020年在中国科学院大学就读硕士研究生,主要研究方向为图计算,大数据处理。
  
\section*{已发表的学术论文:}


\begin{enumerate}[{[}1{]}]
    \item Lei Wang, Liangji Zhuang, \textbf{Junhang Chen}, et al. LazyGraph: Lazy Data Coherency for Replicas in Distributed Graph-Parallel Computation. In PPoPP, 2018. (CCF A类)  
\end{enumerate}



\section*{参加的研究项目及获奖情况:}


\begin{enumerate}[{[}1{]}]
  \item  2017年-2018年,自然基金项目61402445“基于分层图的海量图数据并行编程方法研究”。
  \item  2018年\textbf{优秀学生},计算机体系结构国家重点实验室
  \item  2019年\textbf{优秀学生},计算机体系结构国家重点实验室
\end{enumerate}



\chapter[致谢]{致\quad 谢}\chaptermark{致\quad 谢}% syntax: \chapter[目录]{标题}\chaptermark{页眉}
\thispagestyle{noheaderstyle}% 如果需要移除当前页的页眉
%\pagestyle{noheaderstyle}% 如果需要移除整章的页眉

多年以后,我会经常想起那个在计算所648会议室参加面试的上午。
在这场面试里我第一次见到了日后熟悉的冯晓兵老师,武成岗老师和王蕾老师。
也是这场面试开启了我在计算所参与科研工作的三年研究生岁月。

回顾这三年的研究生岁月,首先要感谢我的指导老师-王蕾老师。
感谢王蕾老师帮我选择了图计算这个充满乐趣和挑战的课题,
更感谢王蕾老师带我参与了发表在国际会议上的论文工作,
让我对学术和科研一窥门径。
王蕾老师总是教我多动脑,多思考动作背后的原理是什么,
讲述代码具体的行为时更要思考这样做的原因。
这样的思考方式使我终身受益。
从小例子着手来解决一个复杂的问题同样也是
我在王蕾老师这里言传身教学到的道理。

计算所是一个充满大师充满机遇的优秀的科研机构。
记得我刚来所里的时候就有幸听到了孙贤和老师的学术报告,后来更是一睹了图灵奖得主John Hopcroft的风采。
无论是学术界最优秀的工作还是产业界最流行的实践,在这里都能听到关于它们最新的解读与分享。

能在这样一个充满学术氛围的地方度过我的研究生岁月要特别感谢我的导师冯晓兵老师。
感谢冯老师给了我在此参与科研的机会,更感谢冯老师在我论文的每个重要阶段给出的宝贵建议和指导。
冯老师组织学生每周一进行的组会极大地开阔了我的视野,
使我了解到了从编译优化到异构编程框架以及深度学习算子优化等各个学术领域的最新进展。
冯老师关于实验数据的敏锐判断和心算能力更是让我深感敬佩。

我在计算所,融科,科一招,青年公寓度过了我难以忘怀的研究生生活。
它们承载了我写代码做实验,饮食休息与运动的快乐。
我在玉泉路,中关村和雁栖湖则上了一些已记不太清但教了我知识的课程。
它们记录了我最后的有时迟到有时认真的学生时代。

最后感谢我的父母家人爱人与朋友,生活因为你们而变得充满希望,我们在彼此的关爱中共同成长。

最后的最后,愿2020我们早日战胜疫情!


\cleardoublepage[plain]% 让文档总是结束于偶数页,可根据需要设定页眉页脚样式,如 [noheaderstyle]
%---------------------------------------------------------------------------%
