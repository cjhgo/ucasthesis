% \chapter{实验结果}

\chapter{总结与展望}

\section{全文总结}

在当今的大数据时代,图作为一种表示和分析大数据的有效方法,在社交网络关系挖掘,推荐系统,网页权重排名,自然语言处理等领域发挥了重要作用。
为此,工业界和学术界开发了很多分布式并行图计算框架。在这些框架中,顶点的副本点扮演了重要作用。
一方面顶点的副本点使得框架能够对单个顶点进行并行处理,另一方面顶点的副本点使得不同机器上的邻居顶点在本地就可以互相访问数据。
然后顶点副本点在带来巨大方便的同时,也引入了副本点之间数据一致性的问题。
在很多图计算框架中,顶点的多个副本点被看作是不可分割的原子部分,副本之间使用急切数据一致性方法进行同步。
这种方法给图计算系统带来频繁的全局同步和通信开销。
针对急切数据一致性的方法存在的问题,我们提出了一种把顶点的副本点看看作是互相独立的点各自独立迭代然后通过基于同样的差值消息进行计算得到相同视图的延迟数据一致性方法。
延迟数据一致性方法有效减少了系统中进行数据一致所需要的数据同步和通信,因而极大地提升了现有分布式图计算的效率。
但是这种方法还缺少一个合适的开启策略来得到相对最优的性能提升。

之前的工作中我们提出了一种基于决策树开启策略的自适应优化方法来解决这一问题,
但是决策树方法依赖于事先的手动调优产生合适的数据集,本质上是一种离线方法,
此外决策树方法对于训练集之外的算法和输入图组合存在失灵的情况。
因此决策树方法无法真正有效地解决延迟数据一致性的自适应优化问题。

在本文中,我们提出了一种基于解的局部性的自适应优化方法,它有效地解决了延迟数据一致方法如何得到相对最优的性能提升的问题。


延迟数据一致方法不同在开启策略下会得到不同程度的性能提升。
为了得到相对最优的性能提升,本文首先对延迟数据一致性方法的性能提升的规律进行了研究。
通过对比观察具体的例子,本文揭示了延迟数据一致性方法的本地计算阶段存在着冗余计算的现象。
冗余计算会增加单次迭代的时间,在图计算过程中带来性能损耗。
通过在具体的算法和输入图上进行试验,
本文发现正是冗余计算造成了延迟数据一致性方法在不同开启策略下得到不同程度的性能提升。
实验发现,那些得到更高程度性能提升的开启策略中冗余计算的比例较低,而性能提升效果不好的开启策略中冗余计算的比例较高。

在发现冗余计算是延迟数据一致性方法性能提升问题的根源之后,
本文进一步从如何减少冗余计算这一角度对延迟数据一致性方法如何得到相对最优的性能提升这一问题展开研究。
最终,本文找到了一种基于解的局部性的自适应优化方法,有效地解决了延迟数据一致方法如何得到相对最优的性能提升的问题。


分布式图计算框架把单个顶点划分为多个副本点。
在延迟数据一致性方法中,每次迭代时,顶点的全局解也就由多个副本点上的局部解构成。
从单个顶点的角度而言,
如果顶点的全局解由多个局部解构成,然后此时在各个局部解所在的副本点上进行延迟数据一致性方法,也就是在本地计算中进行额外的多轮迭代,
由于副本点上都是不准确的局部解,因而此时所进行的本地计算都是在不准确的结果上进行的,所以最终本地计算中更多是冗余计算。
相反,如果顶点的全局解只由某个局部解构成,那么在局部解所在的副本点上开启延迟数据一致性方法,
由于局部解就等于全局解,所以此时相当于在全局同步之后进行的迭代,那么此时所进行的本地计算中冗余计算的比例就较低。
也就是说,从单个顶点而言,当全局解只由某个局部解构成时,本地计算中冗余计算比例较低,有利于开启延迟数据一致性方法,
当全局解由多个局部解构成时,本地计算中冗余计算比例较高,不利于开启延迟数据一致性方法。
在图计算的每次迭代过程中有很多活跃点,这些活跃点中既有全局解由单个局部解构成的,也有全局解由多个局部解构成的。
如果大部分活跃点的全局解只由单个局部解构成,那么此时开启延迟数据一致性能够减少冗余计算,最终得到较好的性能提升。
对于这种情况,我们称之为解的局部性。

在图计算过程中,解的局部性规律是动态变化的。
通过在不同的算法和输入图上进行实验,
我们发现在某些局部性较好的图上,这一规律是图计算一开始就存在的,因而在这些图上,不需要调优,立即开启延迟数据一致性方法就能得到相对最优的性能提升。
而在局部性不好的图上,这一规律在图计算经过数论迭代之后才开始存在,
因而在这些图上,如果立即开启延迟数据一致性方法就会由于开始数论迭代中的大量冗余计算而得到相对较差的性能提升。
相反如果在解的局部性规律开始出现之后再开启延迟数据一致性方法,系统就能得到相对最好的性能提升效果。


在发现图计算过程中顶点上的全局解和局部解存在的局部性规律之后,本文最终提出了一种基于解的局部性的自适应优化方法。
这种方法在线地统计每次迭代时由单个局部解构成全局解的活跃点的数量和比例,从而动态地识别出图计算过程中出现的解的局部性规律。
在识别出活跃点上开始出现解的局部性现象时,优化方法指导系统开启延迟数据一致性方法。
相较于之前提出的决策树方法,这种基于解的局部性的自适应优化方法不需要事先的手工调优和离线训练,同时对于那些决策树方法失灵的情况,这种方法也能正确处理。
最后在不同算法和输入图上的性能评测表明,这种方法指导下的延迟数据一致性方法获得了和手动调优相一致的性能提升效果。

总之,本文的主要贡献可以归结为以下三点:


1. 对延迟数据一致性方法的在不同开启策略下得到不同性能提升这一现象进行了研究,认清了这一方法的性能提升规律。
以具体的例子识别出了延迟数据一致性方法虽然避免了急切数据一致性方法存在的冗余同步和通信,但是引入了新的冗余计算。
并且正是这种冗余计算造成了延迟数据一致性方法在不同开启策略性下得到不同程度的性能提升。

2. 对图计算过程中活跃点上全局解和局部解之间的关系进行了研究,发现了解的局部性这一现象。
通过分析,本课题发现随着图计算的进行,每次迭代中大部分活跃点的全局解只由某个局部解构成,
那么此时局部解所在的副本点完全不需要等待和同步其他副本点就可以继续进行图计算过程。
而这种情况正是有利于减少冗余计算,有利于开启延迟数据一致性方法的情况。

3. 基于解的局部性这一现象,实现了基于解的局部性的自适应优化方法。
这种方法解决了延迟数据一致性方法在不同的输入条件下如何得到相对最优的性能提升这一问题。
相较于之前采用的决策树方法,这种方法不需要事先的离线调优和训练,更不存在某些输入条件下失灵的问题。

\section{未来工作}

本文是对延迟数据一致性方法的延续和补充。
在之前的工作中,我们用理论分析和实验评测证明了延迟数据一致性方法是一种行之有效的提高图计算性能的方法。
在本文的工作中,我们进一步解决了这种方法的自适应优化问题。
通过这两步,延迟数据一致性方法真正成为了一个完善而实用的方法。

在接下来的工作,我们可以进一步将这种延迟数据一致性方法移植到其他分布式图计算框架中来验证这种方法的普适性。


